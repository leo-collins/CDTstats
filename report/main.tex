\documentclass[12pt]{article}

\usepackage{graphicx}
\usepackage{geometry}
\usepackage{amsmath}
\usepackage{amsthm}
\usepackage{amssymb}
\usepackage{float}
\usepackage[parfill]{parskip}

\usepackage[backend=biber, style=numeric-comp, sorting=none]{biblatex}
\addbibresource{refs.bib}

\title{MFC CDT Probability and Statistics Coursework}
\author{Leo Collins}

\begin{document}

\maketitle

\section{Model}

The model will be a two-dimensional linear Gaussian state space model

\begin{equation}
    \mathbf{x}_{t} = \mathbf{A} \mathbf{x}_{t-1} + \mathbf{w}_{t}
\end{equation}
where $\mathbf{x}_{t} \in \mathbb{R}^{2}$ is the state vector at time $t$, $\mathbf{A}$ is the state transition matrix, and $\mathbf{w}_{t} \sim \mathcal{N}(\mathbf{0}, \mathbf{Q})$ is the process noise at time $t$ with covariance matrix $\mathbf{Q}$.
At each time $t$ the state is observed according to
\begin{equation}
    \mathbf{y}_{t} = \mathbf{C} \mathbf{x}_{t} + \mathbf{v}_{t}
\end{equation}
where $\mathbf{y}_{t} \in \mathbb{R}^{2}$ is the observation vector at time $t$, $\mathbf{C}$ is the observation matrix, and $\mathbf{v}_{t} \sim \mathcal{N}(\mathbf{0}, \mathbf{R})$ is the observation noise at time $t$ with covariance matrix $\mathbf{R}$.
\section{Kalman Filter}

\section{Particle Filter}

\section{Results}

\end{document}